\section{Osnovne algebrske strukture}

\subsection{Algebrska struktura}

\subsubsection{Definicija}
Naj bo $S$ poljubna neprazna množica. \\
Vsaki preslikavi $\varphi : S \times S \to S$ rečemo DVOMESTNA NOTRANJA OPERACIJA \\ (ali DNO) na množici $S$. \\[1em]
Sliko urejenega para $(a, b) \in S \times S$ pišemo $a\varphi b$ (namesto običajnega zapisa $\varphi(a, b)$) \\ in jo imenujemo KOMPOZITUM (SESTAV) ELEMENTOV $a$ in $b$ iz $S$. \\[1em]
Dvomestno notranjo operacijo označujemo z znaki: $+, \cdot, \circ, *,\triangle, \heartsuit, \dots$

\subsubsection{Zgled}
\begin{enumerate}[label=\alph*)]
    \item $S = \mathbb{N}$ \\ $\circ$ je običajno seštevanje naravnih števil. \\ $\Rightarrow$ je DNO, saj $\forall a, b \in \mathbb{N}$ je $a \circ b \in \mathbb{N}$.
    \item $S = \mathbb{N}$ \\ $\circ$ je običajno odštevanje naravnih števil. \\ $\Rightarrow$ ni DNO, npr. za $1 \circ 2 = 1 - 2 = -1 \notin \mathbb{N}$.
\end{enumerate}

\subsubsection{Definicija}
DNO $\circ$ na množici $S \ne \emptyset$ je ASOCIATIVNA če za vse elemente $a, b, c \in S$ velja
\begin{align*}
    (a \circ b) \circ c = a \circ (b \circ c)
\end{align*}
KOMUTATIVNA, če za vsaka elementa $a, b \in S$ velja
\begin{align*}
    a \circ b = b \circ a
\end{align*}

\subsubsection{Zgled}
\begin{enumerate}[label=\alph*)]
    \item $S = \mathbb{Z}$ \\ $\circ$ je običajno seštevanje celih števil. \\ $\Rightarrow$ je DNO. \\ $\Rightarrow \circ$ je komutativna, in je asociativna.
    \item $S = \mathbb{Z}$ \\ $\circ$ je običajno odštevanje celih števil. \\ $\Rightarrow$ je DNO.
    \begin{align*}
        a &= 1, b = 0 \\
        a \circ b &= 1 - 0 = 1 \\
        b \circ a &= 0 - 1 = -1   
    \end{align*}
    $\Rightarrow$ ni komutativna.
    \begin{align*}
        a = 1, b &= 2, c = 3 \\
        (a \circ b) \circ c = & (1 - 2) - 3 = -4 \\
        a \circ (b \circ c) = & 1 - (2 - 3) = 2   
    \end{align*}
    $\Rightarrow$ ni asociativna.
    \item $S = \mathbb{R}^{n \times n}$ (kvadratne matrike z realnimi koeficienti) \\ $\circ$ je običajno množenje matrik. \\ $\Rightarrow$ je DNO, ker je rezultat zmnožka spet kvadratna matrika velikosti $n \times n$ z realnimi koeficienti.
    \begin{align*}
        A = 
        \begin{bmatrix}
            a_1 & b_1 \\
            c_1 & d_1 \\
        \end{bmatrix}, 
        B = &
        \begin{bmatrix}
            a_2 & b_2 \\
            c_2 & d_2 \\
        \end{bmatrix},
        C = 
        \begin{bmatrix}
            a_3 & b_3 \\
            c_3 & d_3 \\
        \end{bmatrix} \\
    \end{align*}
    $\Rightarrow$ je asociativno.
    \begin{align*}
        A = 
        \begin{bmatrix}
            1 & 1 \\
            1 & 1 \\
        \end{bmatrix}, 
        B = 
        \begin{bmatrix}
            1 & 1 \\
            0 & 0 \\
        \end{bmatrix} \\[1em]
        A \circ B = 
        \begin{bmatrix}
            1 & 1 \\
            1 & 1 \\
        \end{bmatrix} \cdot 
        \begin{bmatrix}
            1 & 1 \\
            0 & 0 \\
        \end{bmatrix}
        = 
        \begin{bmatrix}
            1 & 1 \\
            1 & 1 \\
        \end{bmatrix} \\[1em]
        B \circ A = 
        \begin{bmatrix}
            1 & 1 \\
            0 & 0 \\
        \end{bmatrix} \cdot 
        \begin{bmatrix}
            1 & 1 \\
            1 & 1 \\
        \end{bmatrix} =
        \begin{bmatrix}
            2 & 2 \\
            0 & 0 \\
        \end{bmatrix}
    \end{align*}
    $\Rightarrow$ ni komutativno.
\end{enumerate}

\subsubsection{Trditev}
Če je DNO $\circ$ na $S \ne \emptyset$ asociativna, potem je produkt (kompozitum) elementov 
$$a_1, a_2, \dots, a_n \in S \text{ } \text{ } \text{ } (n \in \mathbb{N})$$ 
natančno določen z vrstnim redom teh elementov. Tak produkt označimo z
$$a_1 \circ a_2 \circ \dots \circ a_n$$

\paragraph{Dokaz:}
izpustimo!

\subsubsection{Trditev}
Če je $\circ$ asociativna in komutativna DNO na $S \ne \emptyset$, potem je naš produkt elementov
$$a_1, a_2, \dots, a_n \in S \text{ } \text{ } \text{ } (n \in \mathbb{N})$$
enolično določen ne glede na vrstni red naših elementov.

\paragraph{Dokaz:}
izpustimo!

\subsubsection{Definicija}
Naj bo $S \ne \emptyset$ z DNO $\circ$. \\[1em]
Element $l \in S$ je LEVI NEUTRALNI ELEMENT v množici $S$, če za $\forall a \in S$ velja
$$l \circ a = a$$ \\[1em]
Element $d \in S$ je DESNI NEUTRALNI ELEMENT v množici $S$, če za $\forall a \in S$ velja
$$a \circ d = a$$ \\[1em]
Če je $e \in S$ hkrati levi in desni neutralni element v množici $S$, mu rečemo NEUTRALNI ELEMENT. \\[1em]
Oznaka: $(S, \circ)$ ... neprazna množica $S$ z DNO $\circ$.

\subsubsection{Trditev}
Če $(S, \circ)$ premore levi in desni neutralni element, potem sta enaka.

\paragraph{Dokaz:}
Naj bo $l \in S$ levi neutralni element in $d \in S$ desni neutralni element v množici $S$, potem je
$$l = l \circ d = d$$
Torej sklepamo, da je $l = d$, kar smo želeli pokazati.

\subsubsection{Zgled}
\begin{enumerate}[label=\alph*)]
    \item $S = \mathbb{R}^{2 \times 2} = 
    \left\{ 
        \begin{bmatrix}
            a & b \\
            c & d
        \end{bmatrix} 
    ;
    a, b, c, d \in \mathbb{R}
    \right\}$ \\[1em]
    $\circ$ je običajno množenje matrik. \\[1em]
    $\Rightarrow$ je DNO. \\[1em]
    $I = \begin{bmatrix}
        1 & 0 \\
        0 & 1
    \end{bmatrix}$ je neutralni element, saj za $\forall A \in S$ velja $I \cdot A = A \cdot I = A$. \\[1em]
    \item $S = \left\{
        \begin{bmatrix}
            a & b \\
            0 & 0
        \end{bmatrix};
        a, b \in \mathbb{R}
    \right\}$ \\ [1em]
    $\circ$ je običajno množenje matrik.
    \begin{align*}
        \begin{bmatrix}
            a & b \\
            0 & 0
        \end{bmatrix}
        \cdot
        \begin{bmatrix}
            x & y \\
            0 & 0
        \end{bmatrix}
        = 
        \begin{bmatrix}
            ax & ay \\
            0 & 0
        \end{bmatrix}
    \end{align*}
    $\Rightarrow$ je DNO.
    \begin{align*}
        \text{LEVI NEUTRA} & \text{LNI ELEMENT} \\[1em]
        \begin{bmatrix}
            ax & ay \\
            0 & 0 \\
        \end{bmatrix}
        &= 
        \begin{bmatrix}
            x & y \\
            0 & 0 \\
        \end{bmatrix} \\
        &\Downarrow \\
        a = 1, b &= \text{ poljuben } \\
        &\Downarrow \\
        \forall b \in \mathbb{R} \text{ je }
        \begin{bmatrix}
            1 & b \\
            0 & 0 \\    
        \end{bmatrix} 
        \text{ levi } & \text{neutralni element v } S. \\
        &\Downarrow \\
        \text{Levih neutralnih elemen} & \text{tov je neskončno mnogo.}
    \end{align*}

    \begin{align*}
        \text{DESNI NEUTR} & \text{ALNI ELEMENT} \\[1em]
        \begin{bmatrix}
            ax & ay \\
            0 & 0 \\
        \end{bmatrix}
        &= 
        \begin{bmatrix}
            a & b \\
            0 & 0 \\
        \end{bmatrix} \\
        & \Downarrow \\
        ax = a & \Rightarrow x = 1 \\
        ay = b & \Rightarrow y = \frac{b}{a} \\
        & \Downarrow \\
        \text{ Ni OK! Ker } & \text{je odvisno od } a, b. \\
        & \Downarrow \\
        \text{Desni neutralni } & \text{element ne obstaja!}        
    \end{align*}
\end{enumerate}

\subsubsection{Definicija}
Naj bo $(S, \circ)$ premore neutralni element $e \in S$, ter naj bo $a \in S$ poljuben. \\[1em]
Potem $l \in S$ je LEVI OBRAT (ali INVERZ) ELEMENTA $a \in S$ če velja
$$l \circ a = e$$
Element $d \in S$ je DESNI OBRAT ELEMENTA $a \in S$ če velja
$$a \circ d = e$$
OBRAT ELEMENTA $a \in S$ je tak element iz $S$, ki je levi in desni obrat elementa $a$. \\[1em]
Element $a \in S$ je \underline{obrnljiv} (v množici $S$) če premore obrat v množici $S$.

\subsubsection{Trditev}
Naj veljajo oznake iz definicije 1.1.10. 
Neutralni element $e$ je obrat samega sebe.

\paragraph{Dokaz:}
$$e \circ e = e$$

\subsubsection{Trditev}
Naj bo $S \ne \emptyset$ z DNO $\circ$, ki je asociativna in naj bo
$e \in S$ neutralni element. Če ima element $a \in S$ levi in desni obrat v $S$
potem sta enaka.

\paragraph{Dokaz:} Naj veljajo predpostavke iz trditve 1.1.12 in $a \in S$.
$$ \exists \text{ levi obrat za } a \text{ v } S \Rightarrow \exists l \in S: l \circ a = e $$
$$ \exists \text{ desni obrat za } a \text{ v } S \Rightarrow \exists d \in S: a \circ d = e $$
Potem je
$$ (l \circ a) \circ d = e \circ d = d $$
$$ l \circ (a \circ d) = l \circ e = l $$
ker je $\circ$ asociativna operacija. Torej je $l = d$.

\subsubsection{Definicija}
Če je $S \ne \emptyset$ z DNO $\circ$, ki je asociativna, potem rečemo, da je $(S, \circ)$ POLGRUPA.\\
Polgrupa z neutralnim elementom je MONOID. \\
Monoid v katerem je vsak element obrnljiv je GRUPA.
\begin{center}
    \begin{tabular}{|c|c|c|c|}
        \hline
        $(S, \circ)$ & $\circ$ asociativna & $\exists$ neutralen element & $\forall a \in S$ je obrnljiv \\
        \hline
        POLGRUPA & $\checkmark$ & $\times$ & $\times$ \\
        \hline
        MONOID & $\checkmark$ & $\checkmark$ & $\times$ \\
        \hline
        GRUPA & $\checkmark$ & $\checkmark$ & $\checkmark$ \\
        \hline
    \end{tabular}
\end{center}

\subsubsection{Definicija}
Če izbrano DNO na $S \ne \emptyset$ označimo s +, potem govorimo o SEŠTEVAJOČEM \\
(ali ADITIVNEM) ZAPISU. \\[1em]
Element $a + b$ je VSOTA elementov $a, b \in S$, neutralni element označimo z $0 \in S$ \\
(in mu rečemo ničla), obratu elementa $a \in S$ rečemo NASPROTNI ELEMENT in ga \\
označimo z $-a$. \\[1em]
Če izbrano DNO na $S \in \emptyset$ označimo z $\cdot$, potem govorimo o MNOŽEČEM \\
(ali MULTIPLIKATIVNEM) ZAPISU.
$$a \cdot b = ab$$
Element $ab$ je zmnožek (ali PRODUKT) elementa $a, b \in S$, neutralni element označimo \\
z $1 \in S$ (in mu rečemo enka), obratu elementa $a \in S$ rečemo INVERZ, označimo z $a^{-1}$.

\subsubsection{Definicija}
Naj bo $\Omega \ne \emptyset$.
$Map(\Omega) = \{ f : \Omega \to \Omega \}$ $\leftarrow$ množica vseh preslikav iz $\Omega$ v $\Omega$. \\
Množico $Map(\Omega)$ opremimo z (običajno) operacijo levega sestavljanja preslikav:
$$
\forall f, g : \Omega \to \Omega \text{ je } f \circ g : \Omega \to \Omega
$$ 
$$
\text{in } \forall x \in \Omega \text{ velja } (f\circ g)(x) = f(g(x))
$$ \\
Operacija $\circ$ iz definicije 1.1.15 je DNO na $Map(\Omega)$.

\subsubsection{Trditev}
$(Map(\Omega), \circ)$ je monoid.

\paragraph{Dokaz:}
\begin{enumerate}[label=\Roman*)]
    \item $\circ$ je asociativna. (moramo dokazati, oz. dokazano spodaj) 
    $$
    \forall f, g, h \in Map(\Omega) : (f \circ g) \circ h = f \circ (g \circ h)
    $$
    Opazimo:
    \begin{align*}
        ((f \circ g) \circ h)(x) = f(g(h(x))) \\
        (f\circ (g \circ h))(x) = f(g(h(x)))
    \end{align*}
    \item $\exists$ neutralnega elementa v $Map(\Omega)$ za $\circ$
    $$
    \forall x \in \Omega \text{ naj bo } id : x \to x \text{ (identična preslikava)}
    $$
    $$
    \text{Pogazati moramo: } \forall f \in Map(\Omega): f \circ id = id \circ f = f
    $$
    \begin{align*}
        \forall x \Omega \text{ velja: } & (f \circ id)(x) = f(id(x)) = f(x) \\
        & (id \circ f)(x) = id(f(x)) = f(x)
    \end{align*}
\end{enumerate}

\subsubsection*{1.1.15 \space \space Definicija (nadaljevanje)}
Podobno definiramo:
\begin{align*}
    Inj(\Omega) &= \{ f: \Omega \to \Omega ; f \text{ je injektivna}\} \\
    Sur(\Omega) &= \{ f: \Omega \to \Omega ; f \text{ je surjektivna} \} \\
    Bij(\Omega) &= \{ f: \Omega \to \Omega ; f \text{ je bijektivna} \}       
\end{align*}
in jih opremimo z operacijo sestavljanja preslikav z istim predpisom.

\subsubsection{Trditev}
$(Inj(\Omega), \circ)$ in $(Sur(\Omega), \circ)$ sta monoida, $(Bij(\Omega), \circ)$ je grupa.
\paragraph{Dokaz:} D.N. (za domačo nalogo)

\subsubsection{Trditev}








