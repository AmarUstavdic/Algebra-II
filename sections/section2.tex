
\section{Vektorski prostori}

Vektorski prostor, polje $\mathbb{R}, \mathbb{C}, \mathbb{Q}, \mathbb{Z}_p$, $p$ praštevilo, $\mathbb{F}, \mathbb{F}_2, \mathbb{F}_3$.

\subsection{Definicija}
Naj bo $V \ne \emptyset$ z DNO $+ : V \times V \to V$.
Naj bo $\mathbb{F}$ polje in $\cdot : \mathbb{F} \times V \to V$. \\
Algebrska struktura $(V, \mathbb{F}, +, \cdot)$ je VEKTORSKI PROSTOR, če velja:
\begin{vpenumerate}
    \item $\forall u, v, w \in V: (u + v) + w = u + (v + w)$ \\ + je asociativna na množici $V$.
    \item $\forall 0 \in V: \forall v \in V: v + 0 = v = 0 + v$ \\ obstaja neutralni element za +.
    \item $(\exists -v \in V): v + (-v) = 0 = (-v) + v$ \\ vsak element iz množice $V$ ima nasprotni element.
    \item $\forall u, v \in V: u + v = v + u$ \\ + je komutativna operacija na $V$.
    \item $\forall \alpha, \beta \in \mathbb{F}, \forall v \in V: (\alpha \beta)v = \alpha(\beta v)$
    \item $\forall \alpha \in \mathbb{F}, \forall u, v \in V: \alpha(v + u) = \alpha v + \alpha u$
    \item $\forall \alpha, \beta \in \mathbb{F}, \forall v \in V: (\alpha + \beta)v = \alpha v + \beta v$
    \item $\forall v \in V: 1_{\mathbb{F}} \cdot v = v$
\end{vpenumerate}
Rečemo, da je $V$ \underline{vektorski prostor} nad poljem $\mathbb{F}$ za operaciji $+$ in $\cdot$. \\[1em]
Vsakemu elementu iz $V$ rečemo VEKTOR in vsakemu elementu iz polja $\mathbb{F}$ rečemo SKALAR.

\begin{align*}
    + &: V \times V \to V &&\text{ (seštevanje vektorjev)} \\
    \cdot &: \mathbb{F} \times V \to V &&\text{ (množenje skalarja z vektorjem)} \\
    + &: \mathbb{F} \times \mathbb{F} \to \mathbb{F} &&\text{ (seštevanje skalarjev)} \\
    \cdot &: \mathbb{F} \times \mathbb{F} \to \mathbb{F} &&\text{ (množenje skalarjev)}
\end{align*}

\subsection{Zgled}
\begin{enumerate}[label=\alph*)]
    \item 
    \begin{align*}
        &V = \{0\} && 0 + 0 = 0 \\
        &\mathbb{F} \text{ poljubno polje} && \forall \alpha \in \mathbb{F}: \alpha \cdot 0 = \alpha (0 + 0) = \alpha \cdot + \alpha \cdot 0
    \end{align*}
    \begin{align*}
        (\{0\},\mathbb{F}, +, \cdot) \text{ je vektorski prostor.}
    \end{align*}
    $V$ je v.p. nad poljem $\mathbb{F}$  za tako definirani operaciji $+$, $\cdot$. \\
    Rečemo mu TRIVIALNI VEKTORSKI PROSTOR.
    \item 
    \begin{align*}
        &\mathbb{F} \text{ polje, } n \in \mathbb{N} \\    
        &\mathbb{F}^{n} = \mathbb{F} \times \mathbb{F} \times \dots \times \mathbb{F} = \{(a_1, a_2, \dots, a_n); a_1, a_2, \dots, a_n \in \mathbb{F}\} 
    \end{align*}
\end{enumerate}
